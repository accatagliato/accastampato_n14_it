\parstart{L}{a} luce e i fenomeni ad essa associati hanno da sempre affascinato l'uomo e sono stati oggetto di studio per molti scienziati, i quali, nel corso dei secoli, sono riusciti a conferirle un ruolo da protagonista nell'indagine spettroscopica della materia. Negli ultimi cinquant'anni tale ruolo si è via via rafforzato grazie all'invenzione dei dispositivi laser e alla comprensione delle forze ottiche che la luce è in grado di esercitare sulla materia. Tali forze, solitamente dell'ordine del pico-Newton ($10^{-12}~\newton$), possono essere utilizzate per manipolare e investigare oggetti le cui dimensioni possono spaziare da alcuni micron ($10^{-6}~\meter$, ovvero un millesimo di millimetro) fino a pochi nanometri ($10^{-9}~\meter$).

\section*{Un po' di storia}
Il fenomeno dell'intrappolamento ottico e della consequenziale manipolazione ottica di micro e nano particelle è uno dei risultati più affascinanti e di recente scoperta dell'interazione luce-materia.
Keplero, nel 1619, fu il primo a intuire che da tale interazione potesse scaturire una pressione esercitata dalla luce sulla materia, detta pressione di radiazione \cite{lebedew1901untersuchungen,nxcnons1901preliminary}. Egli, infatti, osservando la coda di una cometa, ne spiegò l'origine e la direzione opposta al sole come dovuta alla pressione impressa dai raggi solari sulle particelle sublimate.

Da qui alla realizzazione dei primi dispositivi per la manipolazione ottica della materia passarono diversi secoli, durante i quali è stata formulata la teoria elettromagnetica di Maxwell (1873) per la spiegazione dell'interazione radiazione-materia ed è stata sviluppata la teoria dello scattering di Mie (1908), che descrive matematicamente l'interazione tra un'onda elettromagnetica e una particella sferica. Tale intervallo temporale ha permesso inoltre lo sviluppo delle tecnologie adeguate per la realizzazione dei vari componenti dei setup sperimentali, dal microscopio ai laser.
Sebbene nel 1960 tutti i protagonisti della nostra storia fossero presenti all'appello, si dovettero  attendere altri ventisei anni prima che Arthur Ashkin realizzasse il primo dispositivo a \emph{pinzette ottiche} in grado di manipolare campioni di diversa natura e dimensione \cite{ashkin1970acceleration,ashkin1971optical,ashkin2000history,mcgloin2010forty}.

\section*{Le pinzette ottiche moderne}
%Le pinzette ottiche sono basate su un'estrema focalizzazione di un fascio laser, la quale genera un gradiente d'intensità talmente forte da dare luogo a una forza attrattiva diretta verso il centro della regione focale.
%Una particella (supposta per semplicità sferica e non assorbente) otticamente intrappolata è soggetta a una forza ottica composta principalmente da un contributo repulsivo (scattering) nella direzione di propagazione dal fascio laser e da un contributo attrattivo.

Una particella (supposta per semplicità sferica e non assorbente)  sottoposta a un laser fortemente focalizzato è soggetta a una forza ottica composta principalmente da un contributo repulsivo (scattering) e  uno attrattivo. Il contributo di scattering è dovuto alla pressione di radiazione che spinge la particella nella direzione di propagazione del fascio laser. Il contributo attrattivo invece è dovuto a un'estrema focalizzazione del fascio laser, che genera un gradiente d'intensità talmente forte da dare luogo ad una forza attrattiva diretta verso il centro della regione focale \cite{omori1997observation}.

Una particella si dice quindi \emph{otticamente intrappolata} quando la forza attrattiva, in particolare la componente lungo l'asse di propagazione del laser, risulta maggiore della forza di scattering dovuta alla pressione di radiazione, permettendo la creazione di una trappola ottica stabile, (cfr. \figurename~\ref{fig:ot}).
La particella, per piccoli spostamenti dalla propria posizione di equilibrio, dovuti al suo moto caotico, può essere considerata soggetta ad un potenziale ottico intrappolante di tipo armonico.
Gli spostamenti della particella dalla propria posizione di equilibrio possono essere registrati e analizzati in vari modi, permettendo la ricostruzione del moto caotico tridimensionale degli oggetti all'interno della trappola e la misurazione delle forze agenti su di essi \cite{magazzuoptical}.

\begin{figure}[!b]
\begin{center}
\fsmgraphics[width=.5\columnwidth]{../articles/Magazzu/micro-nano-oggetti/}{ot}
\caption{\textbf{\figurename~1} -- Raffigurazione semplificata di una pinzetta ottica. Le frecce rosse parallele rappresentano il fascio laser  entrante nell'obiettivo, mentre quelle curve rappresentano il fascio focalizzato uscente. La linea curva nera rappresenta il potenziale di tipo armonico al quale è soggetta la particella sferica intrappolata.}
\label{fig:ot}
\end{center}
\vskip-20pt
\end{figure}

\section*{Il potenziale armonico}
In fisica, viene definito oscillatore armonico un qualsiasi sistema che se spostato dalla sua posizione di equilibrio, mostra una forza di richiamo (tendende a riportarlo nel suo precedente stato di equilibrio) proporzionale allo spostamento stesso.
Per semplicità consideriamo una molla, di rigidità $\kappa$, avente un'estremità fissata e un oggetto attaccato all'altra estremità (cfr. \figurename~\ref{fig:ar}). In condizione di riposo ($x=0$), la sua energia potenziale è pari a zero. Non appena spostiamo l'oggetto dalla sua posizione d'equilibrio ($x\neq 0$), la molla accumulerà una certa energia pari a $U=\frac{1}{2}\kappa x^2 $ che darà luogo a una forza di richiamo $F=-\kappa x$  che  riporterà l'oggetto alla sua posizione di equilibrio.

\section*{Applicazioni delle pinzette ottiche}
Fin dalla loro invenzione le pinzette ottiche sono state applicate a diversi campi di ricerca. Il loro successo è dovuto anche al costo relativamente basso di realizzazione, che ne ha permesso la diffusione in centinaia di laboratori in tutto il mondo. Esse hanno rivoluzionato lo studio dei sistemi microscopici, permettendo la manipolazione e l'assemblaggio di biomolecole, cellule, nanostrutture e singoli atomi \cite{marago2013optical}.
Le particelle intrappolate sono generalmente sospese in soluzioni acquose, che permettono l'investigazione e la manipolazione di cellule viventi all'interno del loro ambiente nativo, senza un contatto fisico.
Usando una particella intrappolata come trasduttore di forza è inoltre possibile investigare le interazioni tra particelle e misurare forze con risoluzione del femto-Newton (10$^{-15}~\newton$) \cite{marago2010photonic,rohrbach2005switching}.
L'integrazione delle pinzette ottiche con la spettroscopia Raman ha portato alla realizzazione delle \emph{pinzette Raman}, tramite le quali è possibile intrappolare e investigare le proprietà chimico-fisiche di un solo campione per volta. Tramite tale tecnica è stato possibile ottenere gli spettri Raman dei singoli virus e delle singole cellule in vivo \cite{xie2002near}.
Dall'unione della foto-polimerizzazione e le pinzette ottiche nasce la foto-litografia (Box. 3), grazie alla quale è possibile manipolare e costruire strutture tridimensionali con precisione nanometrica \cite{guffey2010all}, come ad esempio cristalli di zeolite, cioè cristalli in grado di assorbire una data sostanza (acqua, elio, azoto, ...) al di sotto di una certa temperatura e di rilasciarla in fase gassose al di sopra di tale temperatura \cite{woerdemann2010dynamic}.
%Mediante pinzette ottiche è possibile trasferire una certa quantità di momento angolare di spin da fasci laser polarizzati circolarmente a particelle birifrangenti. In tale maniera è possibile realizzare micro e nano macchine, come rotori o pompe all'interno di chips microfluidici.

\begin{figure}[!t]
\begin{center}
\fsmgraphics[width=\columnwidth]{../articles/Magazzu/micro-nano-oggetti/}{armonico}
\caption{\textbf{\figurename~2} -- Rappresentazione del potenziale armonico di una massa collegata a una molla.}
\label{fig:ar}
\end{center}
\vskip-20pt
\end{figure}

\section*{La spettroscopia Raman}
La parola spettroscopia è composta dal termine \emph{spettro} (dal latino \textit{spectrum}, immagine) e dal termine \emph{scopia} (dal greco \textit{skope}, osservazione), e si riferisce all'uso di una sonda, quale ad esempio un fascio laser, per lo studio della struttura e delle proprietà della materia.
Quando un fascio laser con lunghezza d'onda $\lambda_0$ incide sulla materia, viene per la maggior parte diffuso elasticamente, ossia con la medesima lunghezza d'onda ($\lambda=\lambda_0$) per effetto Rayleigh. Tuttavia una piccola percentuale della radiazione incidente subisce una diffusione anelastica: viene cioè diffusa con una lunghezza d'onda $\lambda$ diversa da quella originaria ($\lambda\neq\lambda_0$, effetto Raman). La differenza fra queste lunghezze d'onda è tipica di ciascun materiale, cioè ogni materiale possiede uno spettro Raman caratteristico, come un'impronta digitale. Per questo motivo la spettroscopia Raman rappresenta un valido strumento di riconoscimento e di identificazione dei materiali  non solo in ambito di ricerca ma anche in svariati ambiti applicativi (forense, farmaceutico, beni culturali, ...).

\section*{La foto-litografia}
La foto-litografia è una tecnica che si basa sulla manipolazione ottica degli oggetti e sul fenomeno della foto-polimerizzazione. Un foto-polimero è un polimero (una macromolecola costituita dalla ripetizione di tante unità più piccole chiamate monomeri, come una catena costituita dall'unione delle singole maglie), le cui proprietà cambiano quando viene esposto alla luce ultravioletta ($\lambda \simeq 400\div100~\nano\meter$, dove $1~\nano\meter = 10^{-9}~\meter$). Questi cambiamenti riguardano spesso la struttura interna del materiale e risultano nell'indurimento del materiale stesso.
Mediante la foto-litografia si possono per esempio costruire strutture manipolando vari blocchetti di polimero in modo tale che, una volta posizionati opportunamente, l'intera struttura possa essere illuminata con radiazione ultravioletta per unire i singoli blocchi in un'unica struttura più resistente.  %(un po' come quando facciamo i biscotti dalle forme più strane, mettiamo a contatto i vari pezzi di pasta e poi li inforniamo per unirli e renderli più duri oltre che più cotti e buoni).

Infine, lo sviluppo delle tecnologie ottiche e delle varie tecniche di microfabbricazione ha permesso la costruzione di micro strutture con dei microcanali, sia per farvi scorrere all'interno delle piccolissime quantità di fluido sia la luce. Con tali dispositivi è possibile inviare luce su piccolissime quantità di fluido e raccoglierne lo spettro. In tale maniera è possibile realizzare dei dispositivi spettroscopici aventi dimensioni ridottissime, come quelle di un chip. All'interno di tali chip è possibile realizzare micro e nano macchine come rotori o pompe grazie all'utilizzo delle pinzette ottiche \cite{jones2009rotation}.

\section*{Bibliografia}
\mybibentry{lebedew1901untersuchungen}\\
\mybibentry{nxcnons1901preliminary}\\
\mybibentry{ashkin1970acceleration}\\
\mybibentry{ashkin1971optical}\\
\mybibentry{ashkin2000history}\\
\mybibentry{mcgloin2010forty}\\
\mybibentry{omori1997observation}\\
\mybibentry{magazzuoptical}
% \mybibentry{marago2013optical}\\
% \mybibentry{marago2010photonic}\\
% \mybibentry{rohrbach2005switching}\\
% \mybibentry{xie2002near}\\
% \mybibentry{guffey2010all}\\
% \mybibentry{woerdemann2010dynamic}\\
% \mybibentry{jones2009rotation}

\smallskip
Commenti on-line: \url{http://www.accastampato.it/2015/09/micro-nano-oggetti/}
