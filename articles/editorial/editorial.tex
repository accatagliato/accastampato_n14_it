Le Nazioni Unite hanno indicato il 2015 come Anno Internazionale della Luce, dedicandolo alla forma di energia che più ci è vicina, la luce appunto,
ma che non smette di sorprenderci, con inaspettati fenomeni e applicazioni sempre nuove.
% Sul sito ufficiale tutte le iniziative a livello mondiale: \url{www.light2015.org}.

\smallskip
E pensando alla luce viene subito in mente il Sole, fonte dell'energia che rende \emph{vivo} il nostro pianeta. E il cielo, che conosciamo azzurro, ma amiamo nelle sue maestose sfumature
all'alba e al tramonto. Roberta Caruso ci mostra come ricostruire il ``cielo in una stanza'', anzi, in un bicchiere, spiegando perché il cielo ha il colore che vediamo tutti i giorni.

\smallskip
Che \emph{vediamo}... il mondo sarebbe invisibile se non fossimo dotati naturalmente della più raffinata macchina fotografica concepibile, frutto di milioni di anni di evoluzione.
Antigone Marino ci racconta i molti sforzi che sono stati fatti negli ultimi due secoli per (bio)imitare l'occhio umano, fino agli ultimi sviluppi verso ``l'occhio digitale''.

\smallskip
Ma cosa impressiona la nostra retina o i sensori delle nostre macchine fotografiche? La luce, certo, ma fotoni, onde elettromagnetiche, entrambe le cose? Una domanda che smuove le basi
epistemiologiche del pensiero scientifico, tra teorie, modelli, leggi e principi. Un discorso ampio e complesso, che Marco Valli affronta a partire da un'affermazione tanto semplice, quanto provocatoria:
``l'elettrone non esiste''.

\smallskip
Ma anche se gli elettroni, o in egual modo i fotoni, \emph{non esistono}, nei laboratori ormai se ne creano persino dal nulla, o meglio, dal vuoto! ``Fiat lux'', esclamano Vincenzo Macrì e
Luigi Garziano, mentre ci accompagnano tra scatole vuote e piatti vibranti alla scoperta dell'effetto Casimir.

\smallskip
E cosa farne con tutti questi fotoni? Alessandro Magazzù per esempio li usa come se fossero delicatissime \emph{pinzette} per ``manipolare oggetti minuscoli con la luce'', grazie a una tecnica di recente sviluppo
che ha già trovato una miriade di applicazioni in moltissimi campi, dalla biologia alla scienza dei materiali. Marco Santoro e Santi Scibilia, invece, sono meno delicati e sfruttano ``luce per distruggere,
luce per costruire'' sfruttando laser altamente energetici e il \emph{principio di ablazione}. Antonio Anastasi invece preferisce l'altissima precisione inviando ``luce sul muone''
per misurarne l'anomalia magnetica come mai prima.

\smallskip
E come la natura della luce è doppia, onda e particella, anche questo numero della nostra rivista è doppiamente speciale: oltre a esplorare tutti i segreti della luce, accoglie le firme di giovani ricercatori campani e siciliani,
dalle università di Napoli e di Messina, grazie alle rispettive sezioni della EPS Young Minds, il programma della European Physics Society rivolto ai futuri scienziati d'Europa,
e ai chapter locali della Optical Society of America (OSA) e della Society of Photo-optical Instrumentation Engineers (SPIE).
Un numero che quindi è il frutto della sinergia di tre importanti poli della formazione e della ricerca scientifiche italiane: Messina, Roma e Napoli. Anche la divulgazione della scienza,
come in fondo lo sviluppo stesso del pensiero umano, scientifico e non, unisce e trae forza dall'unione e dall'incontro, dallo scambio di esperienze e punti di vista, dalla sinergia e dalla collaborazione.

\medskip
Buona lettura!

\begin{flushright}
\fsmgraphics[width=2.5cm]{../graphics/}{light2015}
% \caption{Tutte le iniziative a livello mondiale: www.light2015.org.}
\label{fig:light}
\end{flushright}

%\begin{flushright}
%\emph{Alessio Cimarelli}
%\end{flushright}
