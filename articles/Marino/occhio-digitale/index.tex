\title{L'occhio digitale}
\shorttitle{L'occhio digitale}
\subtitle{Evoluzione di una tecnologia che ha preso ispirazione dalla vita: la fotografia}
\suptitle{\phantom{x}}
\articledescr{Le odierne macchine fotografiche sono in grado di gestire la luminosità e il colore come mai prima. Ma se paragonate all'occhio umano, hanno ancora moltissima strada da fare.}
\author{Antigone Marino}
\shortauthor{A. Marino}
\institution{Dipartimento di Fisica, Università degli Studi di Napoli Federico II}
\journalpart{Il Ricercatore Romano}
\articletype{Il Ricercatore Romano}
\stars{2}
\maketitle
\begin{small}
\pichskip{4mm}
\nobalance

\parstart{L}{a} natura è da secoli la più grande fonte d'ispirazione per gli scienziati. D'altra parte è come un laboratorio messo su miliardi di anni fa e che ha dunque avuto il tempo di perfezionare con l'evoluzione i suoi meravigliosi prodotti. Osservando la natura e tentando di imitarla sono nati materiali innovativi, come costumi da bagno ispirati alla pelle squamata degli squali o il velcro nato dalla riproduzione dei semi della bardana maggiore. Nel 1950 il biofisico americano Otto Schmitt conia il termine \emph{biomimetica}, ovvero l'imitazione di modelli naturali per realizzare dispositivi e materiali. L'idea è infatti quella di sfruttare le tecniche affinate dalla natura in secoli di evoluzione per ottimizzare i prodotti dell'uomo, avvalendosi di tecnologie sempre più avanzate ma al momento ancora inadatte a realizzare macchine efficienti come quelle biologiche. Ma è solo il termine a nascere negli anni ‘60, perchè l'uomo in effetti cerca di imitare la natura da molto prima. 
Uno degli esempi più eclatanti di imitazione di un modello naturale è la macchina fotografica. Come l'occhio umano, quest'ultima è stata progettata per catturare immagini, ovvero per vedere. Analizzando il percorso che la luce compie nell'attraversare una compatta o una reflex, ci accorgiamo che altro non è che ciò che accade ogni volta che apriamo le nostre palpebre. Non solo, vedremo più avanti che nel passaggio dai sistemi analogici a quelli digitali tale ispirazione è ancora più marcata.

\section*{Apparato sperimentale}
Iniziamo da una semplice descrizione di quello che uno scienziato chiamerebbe \emph{apparato sperimentale} \cite{mencuccini}. Il primo elemento ottico che la luce incontra sul nostro viso è la cornea, una membrana trasparente convessa che svolge il ruolo di una lente. Dietro di essa si trova l'iride, propriamente un muscolo di colore variabile, con forma e funzione di diaframma. Ha la forma di un disco circolare ed è attraversato dalla pupilla, un'apertura circolare il cui diametro può cambiare proprio grazie ai movimenti dell'iride. Il terzo elemento ottico dell'occhio è il cristallino: una lente che insieme alla cornea consente di mettere a fuoco i raggi luminosi sulla retina. Ha il compito specifico di variare la distanza focale del sistema ottico, cambiando la propria forma, per adattarlo alla distanza dell'oggetto da mettere a fuoco. Infine la luce attraverserà la retina, la membrana più interna del bulbo oculare. È forse la componente fondamentale per la visione essendo formata dalle cellule recettoriali, i coni e i bastoncelli, responsabili di trasformare l'energia luminosa in segnale elettrico che verrà poi trasportato dal nervo ottico al cervello. L'analogia con la macchina fotografica appare quindi ovvia: cornea-lente, iride-diaframma, cristallino-lente, retina-pellicola/sensore.

\begin{figure}[!b]
\begin{center}
\fsmgraphics[width=\columnwidth]{../articles/Marino/occhio-digitale/}{Occhio}
\caption{\textbf{\figurename~1} -- L'occhio umano, la macchina fotografica perfetta.}
\label{fig:occhio}
\end{center}
\vskip-20pt
\end{figure}

Quello che è più interessante da capire è come non solo i componenti di una reflex o anche di una semplice compatta  possono essere tradotti in parti del nostro occhio, ma anche i meccanismi di funzionamento di questi due apparati siano estremamente simili.
La messa a fuoco, ad esempio, in fotografia si realizza spostando degli elementi ottici dell'obiettivo più o meno vicino al sensore, nel bulbo oculare la stessa funzione viene svolta dai muscoli che stirano o comprimono il cristallino, facendogli cambiare forma, e quindi lunghezza focale. 
La regolazione dell'esposizione, che in fotografia si controlla chiudendo o aprendo il diaframma, nell'occhio viene realizzata allo stesso modo dall'iride, un muscolo che contraendosi o rilassandosi, riduce o aumenta la superficie di cristallino esposto ai raggi di luce.

\begin{figure}[!t]
\begin{center}
\fsmgraphics[width=\columnwidth]{../articles/Marino/occhio-digitale/}{Set_up}
\caption{\textbf{\figurename~2} -- Gli elementi costituenti dell'occhio e della macchina fotografica.}
\label{fig:setup}
\end{center}
\vskip-20pt
\end{figure}

\section*{Percezione e bilanciamento del bianco}
I punti dolenti, su cui la tecnologia non è ancora riuscita a fornire un prodotto all'altezza dell'occhio umano sono la gestione della luminosità e del colore. Il semplice motivo è che queste due grandezze fisiche sono legate alla percezione, che è soggettiva da individuo a individuo. Responsabile di tale soggettività è il cervello umano, che per analogia potremmo chiamare il processore della macchina digitale. 
Supponiamo di essere al sole mentre stiamo guardando una parete bianca. L'occhio raccoglierà la luce riflessa dalla parete, il messaggio che arriverà al cervello tramite il nervo ottico è ``questa parete è bianca''.  Se passasse improvvisamente una nuvola, la parete diverrebbe grigia per via della penombra, ma il nostro cervello continuerebbe a vedere la parete bianca perchè effettuerebbe istantaneamente quello che in fotografia si chiama il \emph{bilanciamento del bianco} \cite{freeman}. Se l'oggetto in questione fosse un oggetto a noi sconosciuto, e si trovasse sotto una luce colorata, il nostro cervello non sarebbe  in grado di interpretare il colore vero dell'oggetto, ma lo interpreterebbe in funzione della luce incidente. Solo l'avvicinamento di un oggetto di colore noto, consentirebbe al cervello di bilanciare automaticamente il bianco, estendendo l'interpretazione del giusto colore al primo.
Le macchine fotografiche non godono di questo automatismo. Se fotografiamo una ambiente illuminato con luce al tungsteno, le foto avranno una dominante di colore giallo; se l'illuminazione fosse al neon, la dominante sarebbe azzurra. Nel passaggio dalla fotografia analogica alla digitale si è riusciti a controllare parzialmente questo problema, introducendo dei settaggi standard per il bilanciamento del bianco proprio come ``cielo nuvoloso'', ``luce al tungsteno'', ``luce al neon'', etc. I fotografi professionisti, nel tentativo di imitare l'occhio umano, fotografano un bersaglio di colore grigio prima di iniziare la loro sessione di scatti. Questa foto consentirà di dire alla loro macchina che quel colore, indipendente dalle condizione di luce che lo illuminano, è un grigio. Questo gli consente di mettere un punto fermo nella soggettività della percezione della luce e del colore. Ma è in effetti un palliativo, poiché propriamente valido solo per quel singolo scatto. Il nostro occhio, non solo è rapidissimo nell'aggiornare \emph{per ogni scatto} il bilanciamento del bianco, ma per di più ne può eseguire parecchi contemporaneamente! Vi basti immaginare di essere in una stanza con una luce al tungsteno e una al neon.

\begin{figure}[!b]
\begin{center}
\fsmgraphics[width=\columnwidth]{../articles/Marino/occhio-digitale/}{Bilanciamento_Bianco}
\caption{\textbf{\figurename~3} -- Il bilanciamento del bianco ci consente, indicando al processore che la tavoletta al centro dell'immagine è grigia, di riportare il bianco della maglietta al suo valore reale. Questo procedimento è automatizzato nell'occhio umano.}
\label{fig:bilanciamento}
\end{center}
\vskip-20pt
\end{figure}

\section*{Sensibilità e ISO}
La seconda sostanziale differenza tra occhio umano e fotocamera sta nella sensibilità alla luce. Rispetto al sensore, l'occhio umano non ha una sensibilità uniforme, il che gli permette di catturare molti più dettagli e con una definizione decisamente superiore.
Il sensore della nostra macchina digitale ha una sensibilità uniforme alla luce. In condizioni di scarsa luminosità quello che si fa di solito è aumentare il rapporto tempo di esposizione/numero $f$. Il numero $f$, in fotografia anche detto \emph{numero di stop}, è proporzionale all'inverso del diametro del diaframma. Aumentare il rapporto esposizione/numero $f$ equivale dunque ad aumentando il tempo di esposizione o ridurre gli stop (o equivalentemente ad aumentare il diametro del diaframma). Se ciò non è sufficiente o possibile, possiamo aumentare la sensibilità del sensore incrementando gli ISO. Questo acronimo indica in verità la International Organization of Standardization, la più importante organizzazione europea per la definizione di norme tecniche, il cui equivalente americano è la American Standards Association, da cui ASA, altro acronimo per indicare la sensibilità delle pellicole. Lo standard ISO 5800:1987 definisce due scale (una lineare e una logaritmica) per misurarne il valore. Il sensore trasforma per effetto fotoelettrico la luce che la macchina raccoglie in segnale elettrico. Passando da 100 a 200 ISO, il segnale elettrico generato raddoppia \cite{freeman}. Purtroppo il sensore non solo genera un segnale dalla conversione luce-elettricità, ma anche un segnale di rumore, dovuto per lo più a effetti termici. Questo rumore aumenta all'aumentare della sensibilità. Quindi non sempre è vantaggioso aumentare gli ISO per ottenere migliori immagini.
L'occhio umano, invece, non vede il rumore. Infatti per aumentare la sua sensibilità l'occhio non agisce sul segnale elettrico, bensì sui recettori del segnale luminoso. Aumentando i valori di rodopsina nella retina, l'occhio riesce in pochissimi secondi ad aumentare la sua capacità visiva. Quando si entra in un ambiente a scarsa luminosità questo processo di incremento di rodopsina parte automaticamente, e in circa 30 minuti il nostro occhio sarà in grado di catturare una quantità di luce pari a 600 volte quella che catturerebbe in caso di illuminazione normale. Il tutto senza aggiungere rumore all'immagine.
In termini di ISO, se volessimo provare a convertire la nostra capacità visiva in un numero, potremmo dire che siamo in grado di applicare una moltiplicazione equivalente ISO pari a 60.000 (600 volte 100 ISO) senza aggiungere il benché minimo rumore alle immagini. Le macchine fotografiche più evolute sono in grado di spingersi ben oltre questo valore ma ad un prezzo molto alto: l'immagine è quasi inguardabile già quando si superano gli 8.000 ISO.

\begin{figure}[!t]
\begin{center}
\fsmgraphics[width=\columnwidth]{../articles/Marino/occhio-digitale/}{iso}
\caption{\textbf{\figurename~4} -- Immagini scattate a parità di esposizione variando la sensibilità del sensore.}
\label{fig:iso}
\end{center}
\vskip-20pt
\end{figure}

Questi due punti, bilanciamento e sensibilità, sono quelli su cui tutta la ricerca scientifica si sta concentrando. La macchina fotografica è stata costruita usando come modello proprio l'occhio umano e cercando di replicare ogni sua caratteristica. La tecnologia, nel corso degli anni, ha permesso di fare passi da gigante ma una macchina fotografica è ancora lontanissima dall'imitare pienamente uno strumento così perfetto come l'occhio. 

\balance
\section*{Bibliografia}
\mybibentry{mencuccini}\\
\mybibentry{freeman}

\smallskip
Commenti on-line: \url{http://www.accastampato.it/2015/09/occhio-digitale/}


\vfill
\begin{thebiography}{}%{../articles/Sestili/}
Antigone Marino è ricercatrice CNR presso il Dipartimento di Fisica dell'Università degli Studi di Napoli Federico II. Si occupa di sviluppo e caratterizzazione di materiali per applicazioni telecom. È da sempre appassionata di fotografia, passione per cui ha studiato e lavora nel campo dell'ottica. Dal 2013 coordina il progetto Young Minds della European Physical Society (EPS).
\end{thebiography}
\end{small}

\halfad{sguardi}{-200}
