\title{Luce sul muone}
\shorttitle{Luce sul muone}
\subtitle{Dal Cern al Fermilab: l'esperimento \emph{g-2}}
\suptitle{\phantom{x}}
\articledescr{La luce è un ottimo strumento per le misure ad altissima precisione, ad esempio l'esperimento \emph{g-2}.}
\author{Antonio Anastasi}
\shortauthor{A. Anastasi}
\institution{Dipartimento di Fisica e di Scienze della Terra, Università degli Studi di Messina}
\journalpart{Il Ricercatore Romano}
\articletype{Il Ricercatore Romano}
\stars{3}
\maketitle
\begin{small}
\pichskip{4mm}
\nobalance

\parstart{I}{l} mondo della ricerca, come a esempio quello delle misure ad alta energia o ad altissima precisione, richiede un enorme e continuo sviluppo tecnologico. Se così non fosse, non sarebbe possibile fare nuove scoperte e migliorare quindi, oltre alla conoscenza del mondo fisico da un punto di vista fondamentale, tutti quegli aspetti applicativi che ne conseguono; basti pensare ai progressi della medicina nucleare o alle moderne tecnologie di comunicazione.
Non sempre però in un esperimento i miglioramenti sono dovuti esclusivamente allo sviluppo tecnologico della strumentazione. A volte un semplice effetto della radiazione elettromagnetica (in questo caso della luce visibile) può giocare il suo ruolo fondamentale. È questo il caso dell'esperimento \emph{g-2.}

\begin{figure}[!b]
\begin{center}
\fsmgraphics[width=\columnwidth]{../articles/Anastasi/esperimento-g2/}{precessione}
\caption{\textbf{\figurename~1} -- Una trottola che precede. È ciò che avviene ai muoni immersi in un campo magnetico.}
\label{fig:precessione}
\end{center}
\vskip-20pt
\end{figure}

\section*{Cos'è \emph{g-2}?}
L'esperimento \emph{g-2} è un esperimento nell'ambito della fisica delle particelle che ha come scopo quello di misurare il valore dell'anomalia magnetica del muone $a_{\mu} = \frac{g-2}{2}$ \cite{gminus2}, che indica la deviazione dal valore previsto da Dirac per il fattore giromagnetico (o di Landè) $g=2$. Il punto di forza di questo esperimento non sta nelle grandi energie utilizzate ($3~\giga\electronvolt$ per \emph{g-2}, contro i $13~\tera\electronvolt$ di LHC, il più famoso acceleratore dei giorni nostri costruito al CERN), ma nell'altissima precisione con cui si può ottenere il risultato della misura. Infatti, sia il valore previsto teoricamente per $a_{\mu}$ sia quello sperimentale sono determinati con un altissimo livello di precisione. Ciò comporta che anche piccolissime differenze, dell'ordine della decima cifra decimale, se confermate, possono dirci se la teoria su cui ci basiamo è differente dalla realtà sperimentale. Da ciò si può comprendere come anche in questo caso lo sviluppo tecnologico della strumentazione e degli apparati deve essere notevole per limitare al massimo le fonti di errore.
La misura consiste nel determinare il \emph{momento magnetico anomalo} del muone, ossia come questa particella, esattamente identica all'elettrone tranne che per la sua massa elevata, precede intorno al suo asse (cfr. \figurename~\ref{fig:precessione}).

La storia di questo esperimento ha inizio negli anni '60 al CERN, dove vennero effettuati in tutto tre esperimenti, per poi andare oltre oceano, precisamente ai Laboratori Nazionali di Brookhaven (BNL, New York), fino a giungere al Fermi National Accelerator Laboratory (FNAL, Illinois). Ogni esperimento, per via di miglioramenti tecnologici nell'apparato sperimentale ha portato a ottenere un valore della misura sempre più preciso.

\section*{Luce blu a FNAL}
È nell'ultimo esperimento, in questo momento in costruzione a FNAL, che accanto allo sviluppo tecnologico per migliorare le performance della strumentazione, la luce gioca il suo ruolo fondamentale per migliorare la misura.
Per poter misurare questa proprietà dei muoni è necessario rivelarli tramite degli oggetti chiamati calorimetri, all'interno dei quali i muoni lasciano una traccia in maniera indiretta.
Compito di questi \emph{calorimetri} è quello di trasformare la traccia della particella che li attraversa in un segnale luminoso che può essere rivelato e analizzato. I calorimetri usati dai precedenti esperimenti, detti a \emph{scintillazione}, hanno bisogno di un certo tempo\footnote{Tempo molto piccolo se confrontato con quello a cui siamo abituati, ma rilevante se paragonato alla velocità della luce.} per effettuare questa \emph{trasformazione}. Questo ritardo temporale fa in modo che se due particelle arrivano molto vicine temporalmente, viene persa l'informazione sulla seconda. Questo problema, detto \emph{pile-up}, ha un effetto rilevante sulla precisione della misura.
Per poter risolvere questo problema intrinseco del calorimetro, nel nuovo esperimento si è pensato di utilizzare una proprietà della radiazione elettromagnetica piuttosto semplice e per nulla innovativa: l'effetto Cherenkov (cfr. \figurename~\ref{fig:cherenkov}).
Questo effetto consiste nell'emissione di luce blu da parte di un materiale, il Fluoruro di Piombo in questo caso (PbF$_2$, cfr. \figurename~\ref{fig:PbF2}) quando le particelle che lo attraversano hanno una velocità maggiore di quella che avrebbe la luce se lo attraversasse \cite{Fien, Ach}.
Questa emissione di luce blu, al contrario di ciò che avviene nei calorimetri a scintillazione, è immediata e aiuta a risolvere il problema del ritardo nel trasformare l'informazione della particella in informazione luminosa come nessun altro strumento riesce a fare.

\begin{figure}[!b]
\begin{center}
\fsmgraphics[width=\columnwidth]{../articles/Anastasi/esperimento-g2/}{muone}
\caption{\textbf{\figurename~2} -- Luce Cherenkov emessa dal passaggio di un muone.}
\label{fig:cherenkov}
\end{center}
\vskip-20pt
\end{figure}

\section*{Luce, eterna protagonista}
Non bisogna dimenticare che la luce è comunque di primaria importanza in tutti gli esperimenti, perchè tutti i sistemi di rivelazione non sono altro che grandi occhi, seppur molto più sensibili dei nostri, che devono \emph{vedere} ciò che accade. In questo caso però è la luce stessa, per altro visibile a occhio nudo, che contribuisce direttamente a portare uno sviluppo e un miglioramento a un intero esperimento scientifico.

\begin{figure}[!t]
\begin{center}
\fsmgraphics[width=\columnwidth]{../articles/Anastasi/esperimento-g2/}{PbF2}
\caption{\textbf{\figurename~3} -- Cristallo di PbF$_2$.}
\label{fig:PbF2}
\end{center}
\vskip-20pt
\end{figure}

% \balance
\section*{Bibliografia}
\mybibentry{gminus2}\\
\mybibentry{Fien}\\
\mybibentry{Ach}

\smallskip
Commenti on-line: \url{http://www.accastampato.it/2015/09/esperimento-g2/}


\vfill
\begin{thebiography}{../articles/Anastasi/anastasi}
Antonio Anastasi (\url{antanastasi@unime.it}) si è
laureato presso l'Università degli Studi di Messina nel 2013 con una tesi
sul sistema di calibrazione per l'esperimento g-2 a Fermilab. Dal 2014 è
Dottorando presso l'Università di Messina dove lavora sull'esperimento g-2
in collaborazione con i Laboratori Nazionali di Frascati (INFN) e il
Fermilab.
\end{thebiography}
\end{small}

% \halfad{fryxellsee}{-180}
