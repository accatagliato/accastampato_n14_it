\parstart{{L'}}{elettrone} non esiste. Possiamo dire lo stesso del fotone e del famigerato bosone di Higgs.
Non perché gli scienziati li abbiano \emph{inventati} per farsi beffe di noi, ma semplicemente
perché noi - scienziati compresi - non siamo in grado di \emph{vedere} direttamente queste
particelle: quello che siamo in grado di rilevare, anche con gli strumenti più sofisticati,
sono solo degli \emph{effetti}. Per spiegare, raccontare e comunicare il modo con il quale questi
effetti si manifestano, abbiamo imparato a dare nomi e attribuire caratteristiche alle
cose che ci stanno attorno e sono protagoniste dei fenomeni naturali. Quando le nostre
descrizioni sono utili e ci aiutano a capire meglio il mondo che ci circonda, le manteniamo. A volte,
ci accorgiamo che la nostra spiegazione non è soddisfacente, e dobbiamo rivedere e modificare
il nostro modo di \emph{raccontare} quei fenomeni.
Per affrontare più a fondo la questione,
dobbiamo partire da cosa voglia dire occuparsi di \emph{scienza} e di cosa
comporti l'indagine dei fenomeni che avvengono nell'Universo.

\section*{Le scienze fisiche e naturali}
Le scienze fisiche e naturali si occupano di descrivere e mettere in relazione tra loro i 
fenomeni che accadono nel mondo naturale. Esse procedono attraverso l'osservazione (e la misurazione)
di questi accadimenti.
Affinché le scienze non si riducano a un mero elenco di fatti e numeri, è necessario 
individuare delle regole e delle generalizzazioni che riassumano i comportamenti osservati: 
queste permettono in molti casi di fare delle previsioni, ovvero di poter descrivere in anticipo
lo svolgimento di un evento ancora non accaduto.
Quando lo scienziato non si limita a osservare dei fenomeni naturali, ma ne influenza
direttamento lo svolgimento, si parla di \emph{esperimento}. La pratica sperimentale comporta 
spesso difficoltà notevoli, soprattutto nell'isolare il fenomeno specifico che vogliamo 
studiare da tutti gli altri che gli stanno accadendo attorno e che ne influenzano l'esito. 
Si rendono necessarie delle idealizzazioni e astrazioni, che passano il più delle volte attraverso la formulazione e lo studio di un \textbf{modello}. 
Se il modello è ben formulato e gli esperimenti condotti con successo, lo studio del fenomeno può
dare luogo all'individuazione di una legge fisica, la quale può trovare spazio all'interno 
di una teoria scientifica \cite{encb}.
Il moto delle stelle e dei pianeti sulla volta celeste, ad esempio, presenta molte regolarità. 
Per questo, i popoli dell'antichità avevano costruito dei modelli in grado di prevedere quando 
una certa costellazione sarebbe tornata visibile nel loro cielo. Tuttavia ci vollero molti secoli 
e vari tentativi successivi per arrivare all'ideazione di un modello capace di fornire anche 
una spiegazione del movimento osservato e portare quindi alla formulazione di una teoria 
scientifica (la teoria della gravitazione universale di Newton) capace sia di descrivere fenomeni 
ben noti, sia di prevedere il risultato di un esperimento ancora non realizzato 
(ad esempio il lancio di un satellite nello spazio).

\section*{Teorie, modelli e leggi scientifiche}
Un \emph{modello scientifico} è una rappresentazione semplificata di un fenomeno
troppo complesso per essere analizzato direttamente in tutti i suoi aspetti \cite{bndl}. Per questo 
si usa dire che ``tutti i modelli sono sbagliati, ma alcuni sono utili''. 
In questo contesto ci riferiamo a un \textbf{modello} inteso come un insieme di relazioni 
matematiche che descrivono il fenomeno di interesse, ma le nostre considerazioni valgono anche
pensando alla riproduzione meccanica di un veicolo o di una articolazione 
del corpo umano.
La costruzione e l'analisi dei modelli è uno strumento centrale della scienza: il ragionamento
scientifico passa necessariamente per un modello del mondo che ci circonda. Analizzando
le osservazioni sperimentali e progettando un nuovo esperimento si parte sempre da ipotesi 
che rappresentano già una prima interpretazione di fatti empirici.
Anche nello studio della fisica elementare ci si imbatte presto nei più semplici modelli:
il \emph{punto materiale}, il \emph{pendolo semplice}, il \emph{gas perfetto}. Come si passa quindi da un modello
a una teoria scientifica? Che differenza c'è tra modello e teoria? E quale relazione?
Una \textbf{teoria} scientifica raccoglie elementi che godono di un buon livello di accettazione e 
condivisione all'interno della comunità scientifica. Per questo una teoria generalmente non
può essere espressa in una singola affermazione o formula. Esempi sono la \emph{teoria della relatività}
o la \emph{teoria dell'evoluzione}.
Si noti che spesso, nel parlare comune, si tende a dare al termine \emph{teoria} il significato
di \emph{speculazione} (``ho una mia teoria''), mentre nel mondo scientifico una teoria è qualcosa
di ben assodato e supportato da numerose osservazioni e conferme.
Parliamo invece di \textbf{leggi} (o \textbf{principi}) della scienza in riferimento a 
espressioni (matematiche o verbali) che sintetizzano un singolo concetto, anch'esso ben verificato.
Esempi sono la \emph{legge di conservazione dell'energia} o il \emph{principio di aumento dell'entropia}.
Queste leggi non costituiscono una spiegazione, ma piuttosto una affermazione su \textsl{come} si
svolgono i processi della natura. Non sono scelte dall'uomo e l'uomo non può cambiarle: sono
sempre e in ogni caso basate sull'osservazione dei fenomeni dell'Universo. In questo senso,
non ci garantiscono che un giorno non troveremo casi nei quali la loro validità non sussista,
ma possiamo dire che per quello che conosciamo fino ad oggi esse sono rispettate.
Si tenga presente che la scienza è un processo dinamico. La costruzione delle teorie scientifiche
procede da ciò che è noto a ciò che non è ancora noto:
nuove idee emergono dai risultati degli esperimenti, da analogie di ragionamento tra
fenomeni e contesti diversi. È in questo modo che le teorie scientifiche vengono arricchite o 
superate \cite{hrtm}.

\section*{A cosa servono i modelli?}
In questo processo, i modelli rappresentano uno degli strumenti fondamentali della scienza, per
conoscere l'Universo e imparare come funziona.
Fare scienza vuol dire costruire, testare, confrontare e rivedere modelli \cite{stnf}.
Ci serviamo di modelli, perché la realtà è troppo complicata. In questo senso, dobbiamo stabilire
la relazione tra il nostro modello e il fenomeno che vogliamo capire. L'analisi del modello può
avvenire in modalità differenti: con un esperimento, un esperimento concettuale, attraverso
la soluzione di equazioni o per mezzo di una simulazione numerica (un programma eseguito da un
computer).
In ogni caso, dobbiamo sempre bilanciare le difficoltà insite nell'analisi del modello con
la sua interpretabilità e il suo potere predittivo, quando trasferiamo i risultati
dell'analisi al fenomeno reale. Si rischia di inserire nel modello troppe caratteristiche, 
che poi non riusciamo a identificare nei risultati degli esperimenti. 
Inoltre, dobbiamo sempre ricordare che i nostri modelli rappresentano un tentativo di spiegare
la realtà, alla quale però non appartengono. Ad esempio, ci può capitare di dire che ``i gas reali non 
si comportano secondo il modello del gas perfetto'', ma sarebbe sempre meglio affermare che 
``il modello del gas perfetto non è in grado di descrivere in maniera corretta il comportamento
 dei gas reali per tutti i valori di pressione e temperatura''.
In generale i modelli ci servono per arrivare alla formulazione di una teoria. Tuttavia, si
possono presentare casi nei quali abbiamo già a disposizione una teoria affidabile, la quale però 
risulta troppo complicata da trattare per la descrizione di un particolare fenomeno: 
possiamo allora ricorrere a un
modello per quella applicazione specifica. Ad esempio, l'elettrodinamica quantistica (o \textsl{QED},
\textsl{Quantum ElectroDynamics}) è in linea di principio in grado di descrivere alcune
caratteristiche di un cristallo, ma può risultare molto più semplice affrontare il problema
introducendo ulteriori assunzioni e ragionando in termini meno elementari (ad esempio atomi, orbite 
e legami molecolari).
Infine, a volte può essere utile costruire dei \textsl{toy model} (\emph{modelli giocattolo}),
che non descrivono nessun fenomeno naturale, ma costituiscono degli ottimi banchi di prova
per applicare una teoria e studiare le difficoltà che potrebbero presentarsi nell'analisi di
un caso reale. Si pensi ad esempio al modello di Ising per il ferromagnetismo: nato come
esercizio per studenti, è in seguito divenuto
il riferimento per l'applicazione di numerose tecniche di analisi tra meccanica statistica e
fisica delle alte energie. Invece è bene non confondere i toy model con le semplificazioni utilizzate
per stimare l'ordine di grandezza di una quantità di interesse: le famose \emph{galline sferiche nel vuoto}
delle barzellette!
In definitiva, usiamo i modelli per dare una struttura ai dati osservativi, applicare una teoria 
a un caso specifico o costruire una nuova teoria. Spesso lo studio stimola analogie verso o da 
altri campi di ricerca, mettendo in risalto la grande \textsl{fertilità} di alcuni modelli.
I modelli possono essere usati per fini esplicativi e previsionali, ma a volte anche in semplice
chiave funzionale. Ad esempio, quando parliamo di \emph{apparato circolatorio} nel corpo umano, stiamo
attribuendo una sorta di realtà a se stante alla parte dei costituenti del nostro organismo che
partecipa a una certa funzione. Questo però non vuol dire che possiamo davvero isolare questi aspetti dagli altri, o che cuore, vene e arterie abbiano coscienza di far parte di
un team \emph{particolare} \cite{hczhc}.

\begin{figure}[!t]
\begin{center}
\fsmgraphics[width=\columnwidth]{../articles/Valli/elettrone-non-esiste/}{figura}
\caption{\textbf{\figurename~1} -- Usuali rappresentazioni di un atomo, dei legami atomici in una molecola, di un
tratto di DNA. Fino a che punto questi modelli grafici forniscono una descrizione accurata
della realtà, e in quali aspetti invece sono eccessivamente idealizzati?}
\label{fig:chall_oring_fail}
\end{center}
\vskip-20pt
\end{figure}

\section*{L'elettrone non esiste?}
Siamo ora in grado di riprendere la nostra affermazione iniziale: 
``l'elettrone non esiste''. Come già ricordato, tutto quello che possiamo osservare
e misurare sono gli effetti rilevati dai nostri strumenti. 
Il resto è solo una nostra rappresentazione mentale: elettroni, atomi
e legami chimici costituiscono il modo con il quale raccontiamo e cerchiamo di spiegare
i fenomeni che accadono nell'Universo. Idee e concetti utilissimi, senza dubbio, che ci hanno
permesso di fare sempre più progressi nella nostra comprensione della realtà.
A volte, tuttavia, è proprio questo bisogno di dare una interpretazione in termini di oggetti
vicini alla nostra quotidianità a portarci fuori strada. Alcuni fenomeni richiedono delle
intepretazioni poco intuitive, che sfuggono al senso comune. Tuttavia, esse si sono dimostrate
in grado di farci capire meglio quello che ci accade attorno, ci hanno permesso di fare previsioni 
molto accurate e di arrivare a innovazioni tecnologiche impensabili cento anni fa.
Rappresentare l'elettrone come una \emph{pallina}, probabilmente, ha reso inizialmente più difficile 
accettare il fatto che numerosi fenomeni possano essere spiegati in maniera convincente
assumendo che l'elettrone si comporti come un'onda. Molti concetti che riguardano il mondo microscopico
non ammettono una descrizione in analogia con quella degli oggetti macroscopici (si pensi al concetto 
di spin o al principio di indeterminazione).
Per questi motivi, possono esistere differenti modelli validi dello stesso sistema o processo fisico:
ogni descrizione, necessariamente, si focalizza su un limitato numero di dettagli e aspetti. Ogni
modello comporta delle assunzioni, idealizzazioni o ipotesi semplificative. Dobbiamo, di caso in caso,
capire quanto e cosa possiamo riuscire a spiegare, a fronte delle complicazioni che aggiungiamo
\cite{hczhc}.

\section*{In conclusione...}
...un modello scientifico è una costruzione mentale, un \emph{surrogato} della realtà. 
Come capire, allora, quali caratteristiche del fenomeno studiato è vantaggioso includere nel modello
e quali invece devono essere lasciate da parte? Come si capisce se la descrizione è soddisfacente? 
La chiave è sempre il confronto con la realtà, ovvero con i risultati degli esperimenti. La
fisica è una scienza sperimentale. Non è in grado di \emph{provare} o \emph{dimostrare}. È solo possibile
portare osservazioni a supporto o a confutazione delle nostre teorie. Per questo non esiste 
un unico modello che vada bene per la descrizione di un certo fenomeno, ma solo quello più 
indicato al livello di dettaglio desiderato, in base alle informazioni disponibili.
Fin quando parlare di elettroni, atomi e fotoni ci consentirà di mandare astronauti nello 
spazio e realizzare dispositivi tascabili nei quali immagazzinare la musica di centinaia 
di dischi, vorrà dire che la nostra descrizione ha un senso!
La \emph{verità} è quella che si manifesta e che possiamo mettere al vaglio dell'analisi sperimentale, 
mentre modelli, teorie e leggi scientifiche sono sempre legate a nostre ipotesi, assunzioni e
semplificazioni. In definitiva, rappresentano i nostri tentativi di descrivere in modo sintetico 
l'Universo che ci circonda.

% \balance
\section*{Bibliografia}
\mybibentry{encb}\\
\mybibentry{bndl}\\
\mybibentry{hrtm}\\
\mybibentry{stnf}\\
\mybibentry{hczhc}\\
Corso online `\textsl{Model Thinking}', offerto periodicamente sulla piattaforma \textsl{Coursera}, \url{https://www.coursera.org/course/modelthinking}.\\
Citata nella puntata 1x09 della serie TV `\textsl{The Big Bang Theory}' della CBS. In generale, si veda la voce `\textsl{Spherical cow}' di Wikipedia, all'indirizzo \url{https://en.wikipedia.org/wiki/Spherical_cow}.

\smallskip
Commenti on-line: \url{http://www.accastampato.it/2015/09/elettrone-non-esiste/}
