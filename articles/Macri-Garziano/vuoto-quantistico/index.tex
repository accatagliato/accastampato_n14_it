\title{Fiat lux}
\shorttitle{Fiat lux}
\subtitle{Produzione di luce dal vuoto quantistico}
\suptitle{\phantom{x}}
\articledescr{
%A partire dai filosofi naturalisti greci, ci si è sempre interrogati sull'esistenza del vuoto.
Ai giorni nostri il vuoto come stato stabile di un sistema è un concetto assodato, tanto che sfruttando l'effetto Casimir si può creare luce dal vuoto.}
\author{Vincenzo Macrì, Luigi Garziano}
\shortauthor{V. Macrì, L. Garziano}
\institution{Dipartimento di Fisica, Università degli Studi di Messina}
\journalpart{Il Ricercatore Romano}
\articletype{Il Ricercatore Romano}
\stars{3}
\maketitle
\begin{small}
\pichskip{4mm}
\nobalance

\parstart{B}{asta} puntare alla volta del cielo stellato un comune telescopio per ammirare il più grande spettacolo di cui l'uomo è parte integrante. Sistema solare, nebulose, stelle doppie, supernovae, buchi neri, galassie, ammassi stellari, tutto a portata dei nostri telescopi, tutto a portata della nostra esperienza. 
È sufficiente spostare di qualche grado l'orientazione dello strumento per osservare e fare esperienza di ciò che il senso comune percepisce come \emph{lo spazio vuoto}.

\section*{Il concetto di vuoto secondo il senso comune}
L'assenza di materia ha sempre affascinato i filosofi della Natura. Parmenide e Aristotele sostennero la tesi secondo la quale lo stato di vuoto non è realizzato in Natura, parlando infatti di \emph{horror vacui}, mentre Leucippo e Democrito proposero, nell'ambito dell'atomismo, l'esistenza del \emph{nulla} tra gli atomi costituenti la materia. Con Evangelista Torricelli si apre la fase moderna dello studio del vuoto. Gli esperimenti da lui eseguiti mediante l'utilizzo di tubi di vetro riempiti di mercurio confermarono l'esistenza del vuoto come stato stabile in Natura, risolvendo il dilemma ellenico e aprendo allo stesso tempo la strada a una nuova generazione di quesiti e dubbi sull'effettiva realtà di uno spazio vuoto, incapace di influenzare la dinamica della materia visibile. Seguendo il senso comune lo spazio vuoto ci appare dunque come il contenitore, di dimensione infinita, della materia di cui siamo fatti. 
    
\section*{Le forze di Casimir}
Le teorie Newtoniane considerano lo spazio (al pari del tempo) come una proprietà intrinseca dell'universo, l'intelaiatura di un sistema all'interno del quale solo le parti della materia visibile interagiscono tra loro. Successivamente, grazie agli studi sperimentali sull'elettromagnetismo ad opera di Michael Faraday, lo spazio vuoto venne identificato come il mezzo di propagazione dell'interazione elettromagnetica. Due particelle cariche elettricamente interagiscono scambiando energia con il campo elettromagnetico da esse generato, il quale si propaga attraverso lo spazio così come un'onda si propaga sulla superficie del mare. Infine la teoria della relatività generale di Einstein dimostra che lo spazio vuoto non solo è strettamente connesso alla materia in esso contenuta, ma ne è anche influenzato:
lo spazio si curva in presenza di elevate quantità di materia e la forza di gravità stessa è una manifestazione di questa curvatura.

\begin{figure}[!b]
\begin{center}
\fsmgraphics[width=.5\columnwidth]{../articles/Macri-Garziano/vuoto-quantistico/}{img1}
\caption{\textbf{\figurename~1} -- Rappresentazione grafica dell'esperimento ideale che dimostrerebbe l'effetto Casimir Dinamico. Lo specchio oscilla adiabaticamente intorno al suo punto di equilibrio a velocità relativistiche. In questa situazione il campo elettromagnetico di punto zero viene pertubato, permettendo così la conversione dei fotoni virtuali in reali.}
\label{fig:Im1}
\end{center}
\vskip-20pt
\end{figure}

Lo sviluppo della meccanica quantistica  agli inizi del secolo scorso portò a una riconsiderazione dello stato di vuoto. Secondo il principio di indeterminazione di Heisenberg non è possibile determinare contemporaneamente la posizione e la velocità di una particella con precisione arbitraria. Detto diversamente, se di una particella conosco con alta precisione la sua posizione, ne consegue che la velocità può essere nota solamente con bassa precisione, e viceversa. Il principio di indeterminazione di Heisenberg vieta dunque l'esistenza di uno stato di assoluta immobilità e pertanto a ogni sistema fisico deve essere attribuito un contenuto minimo non nullo di energia, la cosiddetta energia di punto zero. Il principio vale in generale e per tutte le grandezze accoppiate, incluse energia e tempo.
Si possono avere fluttuazioni del valore dell'energia via via maggiori se si considerano tempi sempre più piccoli.
Siamo quindi arrivati al concetto di fluttuazioni del vuoto. Come ebbe modo di affermare J.C. Maxwell, il padre del moderno elettromagnetismo, il vuoto è ciò che rimane una volta tolto tutto il possibile.
Quant'è l'energia ad esso associata? Se fosse zero questo sarebbe in contraddizione con il principio di Heisenberg, in quanto saremmo in presenza di una grandezza il cui valore è definito con precisione infinita. Dal momento che questo non è possibile, si deve ammettere l'esistenza di fluttuazioni di energia e la conseguente creazione spontanea, anche se per istanti molto brevi, di energia. Tale energia si manifesta come coppie particella-antiparticella dette \emph{particelle virtuali}, oppure come onde elettromagnetiche, dette \emph{fotoni virtuali}.
Abbiamo quindi una nuova definizione: il vuoto quantistico, ossia lo stato di minima energia di un sistema. Questo vuoto ha delle proprietà che possono essere studiate sperimentalmente. L'effetto Casimir \cite{Casimir} consiste nelle fluttuazioni del vuoto elettromagnetico associate ai fotoni virtuali. La presenza di corpi conduttori in una regione di spazio limita il numero e il tipo dei fotoni virtuali che si possono manifestare. A causa di questo, due piatti conduttori paralleli, posti a distanza ravvicinata, sono soggetti ad una forza attrattiva (detta \emph{forza di Casimir}) dovuta alle differenti pressioni di radiazione dei modi elettromagnetici virtuali.

\begin{figure}[!t]
\begin{center}
\fsmgraphics[width=\columnwidth]{../articles/Macri-Garziano/vuoto-quantistico/}{img23}
\caption{\textbf{\figurename~2} -- In alto, vignetta del Casimir Dinamico: una scatola che idealmente racchiude il vuoto viene agitata da un ometto, simulando così l'aspetto dinamico dell'effetto Casimir, e dunque la conversione dei fotoni virtuali in reali. Si osserva la fuoriuscita della luce all'apertura della scatola.
In basso, vignetta del Casimir Spontaneo: La scatola che idealmente racchiude il vuoto, ha potenzialmente la capacità di convertire al suo interno fotoni virtuali in fotoni reali se lasciata libera di \emph{decadere} liberamente in uno stato energetico più basso.}
\label{fig:Im2}
\end{center}
\vskip-20pt
\end{figure}

% \begin{figure}[!t]
% \begin{center}
% \fsmgraphics[width=\columnwidth]{../articles/Macri-Garziano/vuoto-quantistico/}{img2}
% \caption{\textbf{\figurename~2} -- Vignetta del Casimir Dinamico: una scatola che idealmente racchiude il vuoto viene agitata da un ometto, simulando così l'aspetto dinamico dell'effetto Casimir, e dunque la conversione dei fotoni virtuali in reali. Si osserva la fuoriuscita della luce all'apertura della scatola.}
% \label{fig:Im2}
% \end{center}
% \vskip-20pt
% \end{figure}
% 
% \begin{figure}[!b]
% \begin{center}
% \fsmgraphics[width=\columnwidth]{../articles/Macri-Garziano/vuoto-quantistico/}{img3}
% \caption{\textbf{\figurename~3} -- Vignetta del Casimir Spontaneo: La scatola che idealmente racchiude il vuoto, ha potenzialmente la capacità di convertire al suo interno fotoni virtuali in fotoni reali se lasciata libera di \emph{decadere} liberamente in uno stato energetico più basso.}
% \label{fig:Im3}
% \end{center}
% \vskip-20pt
% \end{figure}

\section*{Effetto Casimir dinamico e spontaneo}
Ci furono diversi tentativi sperimentali per la verifica dell'effetto Casimir, ma fu ben chiaro da subito che ciò sarebbe stato molto difficile, in quanto si trattava di portare dei corpi estesi a distanze micrometriche. La prima chiara verifica sperimentale è di S. K. Lamoreaux \cite{Lamoreaux}, che misurò la forza fra un piatto piano e una superficie sferica. Solo alcuni anni più tardi fu realizzata la misura della forza di Casimir nella configurazione originale a facce piane parallele, dunque senza l'uso di una superficie sferica. Grazie a questo successo si intraprese una nuova linea di ricerca concernente il vuoto in presenza di superfici che però non stanno più ferme ma si muovono oscillando a frequenza elevatissima. In questo nuovo esperimento si studia il cosiddetto \emph{effetto Casimir dinamico} nel quale la presenza di una parete oscillante in interazione con i fotoni virtuali rende possibile la creazione di coppie di fotoni reali, cioè luce. In realtà non vi è nessuna estrazione di energia dal vuoto, in quanto l'energia spesa per far muovere la parete è estremamente maggiore di quella che si pensa di ricavare sotto forma di luce. Anche in questo caso, però, il fenomeno si capisce solo pensando alla presenza di fotoni virtuali che permeano tutto lo spazio. L'effetto Casimir dinamico è stato recentemente confermato sperimentalmente, è stato possibile cioè estrarre i fotoni dal loro stato virtuale e trasformarli in fotoni reali. Anziché utilizzare uno specchio convenzionale i fisici sperimentali hanno sfruttato l'analogia con la Circuit QED (circuiti elettrici in elettrodinamica quantistica), nella fattispecie è stata usata una linea di trasmissione connessa a un dispositivo che fa da interfaccia quantistica superconduttiva chiamato SQUID (\emph{superconducting quantum interference device}). Questo dispositivo può essere usato per cambiare l'effettiva lunghezza elettrica della linea di trasmissione, e questo cambiamento equivale all'effetto del movimento di uno specchio elettromagnetico nel vuoto.
Modificando molto rapidamente la direzione del campo magnetico  gli scienziati
sono stati in grado di fare vibrare lo specchio ad una velocità pari a circa il 25\% della velocità della luce, riuscendo così a generare un numero non trascurabile di fotoni dallo stato di vuoto \cite{wilson}. Recenti studi teorici hanno dimostrato che, in contrasto col Casimir dinamico, è possibile generare coppie di fotoni reali senza l'applicazione di forze esterne al sistema. Ciò è stato possibile nell'ambito della Circuit QED in cui vengono utilizzati atomi artificiali accoppiati a guide d'onda superconduttrici. Si parla in questo caso di \emph{Casimir spontaneo} \cite{stassi}.

\balance
\section*{Bibliografia}
\mybibentry{Casimir}\\
\mybibentry{Lamoreaux}\\
\mybibentry{wilson}\\
\mybibentry{stassi}

\smallskip
Commenti on-line: \url{http://www.accastampato.it/2015/09/vuoto-quantistico/}


\vfill
\begin{thebiography}{}%{../articles/Marrocchio-Tartaglia/tartaglia}
Vincenzo Macrì (\url{vmacri@unime.it}) si è laureato presso l'Università degli Studi di Messina nel 2014. Attualmente è dottorando presso lo stesso ateneo e fa parte dell'associazione studentesca EPS Young Minds Messina.

Luigi Garziano (\url{lgarziano@unime.it}) si è laureato presso l'Università degli Studi di Messina nel 2012.
Attualmente è dottorando presso lo stesso ateneo e di recente ha trascorso un mese al RIKEN Wako Center (Giappone) per una collaborazione scientifica.
\end{thebiography}
\end{small}

% \halfad{fryxellsee}{-180}
