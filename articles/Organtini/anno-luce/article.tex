\parstart{S}{olo} dalla seconda metà del 2014 la Facoltà di Scienze dell'Università Sapienza di Roma si è dotata della figura istituzionale del Delegato alla Comunicazione Scientifica nella persona del sottoscritto. Quale migliore occasione della concomitante proclamazione del 2015 come Anno Internazionale della Luce per mettere alla prova la capacità della Facoltà di organizzare eventi di diffusione dei risultati della ricerca attraverso questo nuovo ufficio?
È così che, dopo alcune riunioni con i diversi Dipartimenti interessati, abbiamo deciso di dar vita a una grande e originale mostra sul tema della luce da tenersi nel corso del 2015.

\section*{Dalla Notte dei Ricercatori alla Maker Faire}
La mostra sarà inaugurata in occasione della Notte dei Ricercatori, il 25 settembre, con un'apertura serale straordinaria. In quell'occasione saranno presenti tutti i ricercatori che hanno contribuito per fare da guida, nonché tutti coloro che in qualche maniera avranno contribuito alla sua realizzazione.

Nelle settimane successive la mostra sarà aperta su prenotazione per le scuole, fino al 16 ottobre, quando nella nostra Università avrà inizio l'edizione europea della Maker Faire 2015: un evento di portata colossale per il quale si prevede, nei tre giorni di apertura, un afflusso di pubblico straordinario valutato nell'ordine delle centomila persone. Durante i tre giorni della Maker Faire, e
nei giorni successivi, fino alla chiusura prevista per il 21 ottobre, la mostra sarà visitabile da tutti liberamente.

Un'occasione così importante richiedeva di pensare a qualcosa di straordinario: così abbiamo pensato di costruire un percorso didattico guidato dalla fisica della luce, attorno al quale costruire una serie di \emph{deviazioni} verso discipline diverse. La mostra si aprirà con una sezione sull'ottica geometrica nella quale il visitatore avrà modo di sperimentare i fenomeni luminosi spiegabili alla
luce della teoria corpuscolare della luce. Il visitatore, successivamente, sarà testimone dei fenomeni difettivi che fanno ritenere che la luce sia, in realtà, un'onda. In una terza sezione, infine, l'esperimento dell'effetto fotoelettrico dimostrerà al visitatore che la luce non si può interpretare né come onda né come flusso di particelle, ma come entrambe le cose allo stesso tempo, introducendo così la meccanica quantistica.

% \section*{Non solo fisica}
La fisica non è l'unica scienza presente: per la biologia si mostrerà come la luce influenzi lo sviluppo di organismi sia nel regno vegetale che in quello animale; per la mineralogia verrà dimostrato come la luce polarizzata consenta l'identificazione delle rocce e per la matematica saranno esposti modelli in 3D ricostruiti a partire da immagini bidimensionali, con tecniche che
impiegano le leggi dell'ottica.

Anche la fruibilità della mostra avrà un carattere innovativo: in questo percorso non troverete i poster con la spiegazione dettagliata di quel che vedrete. I poster si limiteranno a illustrare brevemente cosa state osservando e cosa ci si aspetta che facciate. La presenza di QR-code, inquadrabili con un comune smartphone, vi permetterà quindi di accedere a contenuti di livello
differenziato per target: bambini, studenti, docenti, pubblico generico. Dopo aver visitato la mostra il pubblico disporrà così di una raccolta di link dai quali attingere tutte le informazioni necessarie per comprendere ciò che ha visto.

Le installazioni interattive permetteranno ai visitatori di eseguire gli esperimenti in prima persona, ma in sicurezza e con la garanzia di corretta esecuzione. E per la prima volta, le installazioni non saranno realizzate da professionisti del settore, ma da maker volontari della Fondazione Mondo Digitale di Roma, con la quale abbiamo stipulato un accordo. È già all'opera un team di esperti e
motivatissimi maker che sta cominciando a realizzare installazioni di grande livello tecnico e artistico. Sì, perché tutto il percorso della mostra è pensato per essere curato anche dal punto di vista del design e sarà anche corredato di specifiche installazioni artistiche. Le riproduzioni di alcune opere d'arte che richiamino i fenomeni illustrati nel percorso mostreranno come arte e scienza non siano affatto due campi del sapere diversi, ma che si influenzano a vicenda e sono entrambi parte integrante e fondamentale del progresso dell'umanità.

% \balance
% \section*{Bibliografia}
% \mybibentry{}\\
% \mybibentry{}

\smallskip
Commenti on-line: \url{http://www.accastampato.it/2015/09/millumino-di-scienza/}
