\title{Luce per distruggere\\luce per costruire}
\shorttitle{Luce per distruggere e costruire}
\subtitle{Distruggere un materiale per crearne uno nuovo con proprietà totalmente differenti}
\suptitle{}
\articledescr{Grazie alla luce laser è possibile distruggere un materiale per crearne uno nuovo con proprietà totalmente differenti.}
\author{Marco Santoro, Santi Scibilia}
\shortauthor{M. Santoro, S. Scibilia}
\institution{Dipartimento di Fisica, Università degli Studi di Messina}
\journalpart{Il Ricercatore Romano}
\articletype{Il Ricercatore Romano}
\stars{2}
\maketitle
\begin{small}
\pichskip{4mm}
\nobalance

\parstart{L}{a} scienza dei materiali è un campo di ricerca che ha portato ad un notevole sviluppo tecnologico negli ultimi decenni.
In quest'ambito si definiscono \emph{nanostrutture} quei materiali in cui le dimensioni sono dell'ordine dei $\nano\meter$ (di solito si parla di dimensioni che spaziano dai $10^{-7}$ ai $10^{-9}~\meter$).
È interessante lo studio di questo tipo di materiali perché le loro dimensioni ridotte forniscono proprietà che essi non possiederebbero nella loro forma \emph{tridimensionale} \cite{Vajtai,Lindsay}.
L'approccio che solitamente si utilizza in fisica per la realizzazione di materiali nanostrutturati è di tipo \emph{top-down} che, partendo da strutture di dimensioni macroscopiche, permette di arrivare tramite raffinamenti successivi a materiali le cui dimensioni raggiungono anche i pochi nanometri.

\begin{figure}[!b]
\begin{center}
\fsmgraphics[width=\columnwidth]{../articles/Santoro-Scibilia/ablazione-laser/}{setup}
\caption{\textbf{\figurename~1} -- Schema del setup sperimentale utilizzato per l'ablazione da fasci laser impulsato in liquidi.}
\label{fig:setup}
\end{center}
\vskip-20pt
\end{figure}

\section*{Come distruggere, come creare}
Uno dei metodi più versatili per la realizzazione di questo tipo di materiali è la \emph{Pulsed Laser Deposition} che sfrutta l'utilizzo della luce di un laser impulsato altamente energetico che viene focalizzato su un campione. È lo stesso principio che ci permette, focalizzando la luce del sole con una lente d'ingrandimento, di bruciare un pezzo di carta, con la differenze che l'intensità del laser è più elevata. Si parla infatti di densità superficiali di energia che vanno dai pochi $\joule\per\centi\meter\squared$ alle centinaia di $\joule\per\centi\meter\squared$ e di un tempo di rilascio molto più breve, solitamente su scale temporali che vanno dai $10^{-9}$ ai $10^{-12}~\second$). 

L'energia del laser viene assorbita e convertita in energia termica innescando il \emph{meccanismo di ablazione} che porta all'espulsione di materiale dal campione producendo un micro-plasma costituito da atomi, ioni ed elettroni. Il meccanismo di ablazione si attiva perchè l'energia trasferita dall'impulso laser è molto alta, così come la temperatura raggiunta dal campione nel punto in cui viene focalizzato l'impulso e, da un punto di vista cinetico, viene favorito il processo di espulsione delle particelle rispetto a quello di diffusione del calore all'interno del campione \cite{Chrisey}. Queste particelle, una volta espulse, si riaggregano (in ambiente gassoso o in liquido) formando degli aggregati (cluster) di dimensioni variabili che vanno dai pochi \emph{nanometri} fino anche a qualche frazione di \emph{micrometro}, in base alle condizioni sperimentali che vengono utilizzate per la loro sintesi.

\begin{figure}[!b]
\begin{center}
\fsmgraphics[width=\columnwidth]{../articles/Santoro-Scibilia/ablazione-laser/}{sem}
\caption{\textbf{\figurename~2} -- Immagine di un substrato di nanoparticelle realizzata utilizzando un microscopio elettronico a scansione che rende possibile un'analisi morfologica delle strutture create.
%\footnote{Un microscopio elettronico a scansione utilizza un fascio di elettroni che viene accelerato e focalizzato sulla superficie del campione. I processi di interazione elettrone-campione generano dei segnali (elettroni a basse energie, raggi X, catodoluminescenza, ecc..) che vengono acquisiti da opportuni detector e successivamente elaborati per costituire un'immagine.}
}
\label{fig:sem}
\end{center}
\vskip-20pt
\end{figure}

\section*{Cosa creare e perché}
Le nanoparticelle così realizzate possono essere utilizzate per svariate applicazioni: dalla nanoelettronica alla farmaceutica, dall'aerospaziale alle nanobiotecnologie. In questo campo, un'applicazione di particolare interesse è quella del \emph{drug delivery} \cite{Mora}, riguardante la possibilità di inserire farmaci all'interno di nanocapsule, costituite spesso da polimeri, assieme a nanoparticelle metalliche, formando quindi un sistema nanocomposito. In questo modo è possibile sfruttare le proprietà magnetiche delle nanoparticelle (in particolare di ossidi metallici) per veicolare il moto delle nanocapsule mediante un campo magnetico. È possibile inoltre, sfruttando le capacità di assorbimento ottico possedute da nanoparticelle di metalli nobili (come oro o argento), aumentare la percentuale di rilascio del farmaco quando il sistema viene sottoposto ad una sollecitazione ottica di opportuna lunghezza d'onda. Per essere utilizzabili, è necessario che questi nanocompositi rientrino nell'intervallo di biocompatibilità imposto dal sistema sanitario nazionale; è quindi possibile inserirli in un organismo e far avvenire un rilascio localizzato e controllato del farmaco. Questo tipo di sistemi viene utilizzato spesso per il trattamento di tumori perchè permette la realizzazione di una terapia non invasiva.

Si vede quindi come, attraverso l'utilizzo della luce laser sia possibile letteralmente distruggere un materiale per poter creare da questo delle nuove strutture (le nanoparticelle) con proprietà meccaniche, ottiche ed elettriche completamente diverse dal materiale di partenza. La luce ci permette quindi di modificare radicalmente il comportamento dei materiali.

\balance
\section*{Bibliografia}
\mybibentry{Vajtai}\\
\mybibentry{Lindsay}\\
\mybibentry{Chrisey}\\
\mybibentry{Mora}

\smallskip
Commenti on-line: \url{http://www.accastampato.it/2015/09/ablazione-laser/}


\vfill
\begin{thebiography}{}%{../articles/Sestili/}
Marco Santoro (\url{santoro.marco92@gmail.com}) si è laureato in Fisica presso
l'Università degli Studi di Messina nel 2015 con una tesi nell'ambito dei nanomateriali.
Vincitore di un concorso di Dottorato di Ricerca presso lo stesso ateneo nel 2015.
Dal 2014 fa parte dell'associazione studentesca EPS Young Minds di Messina.

Santi Scibilia (\url{sscibilia@unime.it}) si è laureato presso l'Università degli Studi di Messina nel 2011, dove ha anche conseguito il titolo di Dottore di Ricerca nel 2015 con una tesi sulle nanostrutture prodotte mediante ablazione laser. Dal 2013 fa parte dell'associazione studentesca EPS Young Minds di Messina.
\end{thebiography}
\end{small}

\halfad{futuro-remoto}{280}
