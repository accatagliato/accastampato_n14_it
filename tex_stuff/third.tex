Accastampato non è un periodico, pertanto non è registrato e non ha un direttore responsabile. 
È un esperimento di comunicazione realizzato dall'associazione Accatagliato degli studenti di fisica 
di Roma con il duplice obiettivo di mostrare al pubblico non specialistico e 
agli studenti delle scuole superiori le ricerche portate avanti nell'area romana e di fornire l'occasione agli studenti 
universitari e ai giovani ricercatori di raccontare il proprio lavoro quotidiano e di confrontarsi 
con la comunicazione scientifica non specialistica.

\bigskip
La rivista è prodotta dal motore di composizione tipografica \LaTeX.
Si ringraziano per questo Donald E. Knuth, Leslie Lamport, 
il \TeX\ Users Group (\url{www.tug.org}) e Gianluca Pignalberi.
I sorgenti sono sviluppati e mantenuti da Alessio Cimarelli e sono disponibili su Github:
\url{https://github.com/accatagliato/accastampato_n14_it}.

\bigskip
Impaginazione: Alessio Cimarelli\\
Copertina: Silvia Mariani (immagine di Mato Rachela)

% \bigskip
% Per la traduzione in italiano si ringraziano Laura Caccianini, Massimo Margotti, Martina Pugliese, Alessio Cimarelli.

\vspace{1cm}
\textit{Gli articoli contenuti in questo numero sono protetti con marca digitale grazie a \url{patamu.com}}
\begin{center}\includegraphics[width=0.3\textwidth]{patamu}\end{center}

\vspace{1cm}
\textit{Quest'opera è rilasciata sotto la licenza Creative Commons \emph{Attribuzione-Non commerciale-Condividi allo stesso modo}
4.0 Unported. Se non specificato altrimenti, tutti gli articoli in essa contenuti sono rilasciati dai rispettivi autori 
sotto la medesima licenza. Per leggere una copia della licenza visita il sito web
\url{http://creativecommons.org/licenses/by-nc-sa/3.0/} o spedisci una lettera a Creative Commons, 171 Second Street, 
Suite 300, San Francisco, California, 94105, USA.}
\begin{center}\includegraphics[width=0.3\textwidth]{by-nc-sa}\end{center}
\vspace{1.5cm}
