%Accastampato non è un periodico, pertanto non è registrato e non ha un direttore responsabile. È un esperimento di comunicazione realizzato dall'associazione Accatagliato degli studenti di fisica dell'Università Sapienza di Roma con il duplice obiettivo di mostrare al pubblico non specialistico e agli studenti delle scuole superiori le ricerche portate avanti nell'area romana e di fornire l'occasione agli studenti universitari e ai giovani ricercatori di raccontare il proprio lavoro quotidiano e di confrontarsi con la comunicazione scientifica non specialistica.

%\vspace{1cm}
La rivista è disponibile on-line e come app per iPad e iPhone, è navigabile sia da computer che da cellulare ed è scaricabile nei formati PDF ed ePUB.
\begin{center}
\includegraphics[width=0.3\textwidth]{accastampato_qrcode}\\
\url{http://www.accastampato.it}\\
\bigskip
\includegraphics[width=0.3\textwidth]{app_qrcode}\\
\end{center}

% Questa rivista è disponibile anche in italiano.
% This magazine is also available in English.
% Ce magazine est également disponible en français.
% Esta publicación también está disponible en español.

\bigskip
I lettori possono esprimere commenti o fare domande agli autori on-line sulle pagine dedicate ai singoli articoli.
I QR Code che corredano alcuni articoli codificano gli URL di pubblicazione on-line e sono generati mediante \url{invx.com}

%\medskip
%La rivista è prodotta dal motore di composizione tipografica \LaTeX.
%I sorgenti sono sviluppati e mantenuti da Alessio Cimarelli e sono disponibili richiedendoli alla Redazione.

%\medskip
%Impaginazione: Alessio Cimarelli\\
%Copertina: ...

% \vspace{1cm}
% \textit{Gli articoli contenuti in questo numero sono protetti con marca digitale grazie a \url{patamu.com}}
% \begin{center}\includegraphics[width=0.3\textwidth]{patamu}\end{center}
% 
%\vspace{1cm}
%Questo numero della rivista è realizzato con la collaborazione della Commissione Europea
%e dell'Associazione Frascati Scienza in occasione della Settimana della Scienza 2012 e
%della Notte Europea dei Ricercatori 2012
%\begin{center}\fsmgraphics[height=1cm]{}{logo_ue}~\fsmgraphics[height=1cm]{}{frascati_scienza}\end{center}

\vspace{1cm}
Accastampato è realizzato con il patrocinio
del Dipartimento di Fisica dell'Università Sapienza di Roma, % (\url{http://www.phys.uniroma1.it/}),
del CNR Istituto dei Sistemi Complessi (ISC), Unità Sapienza di Roma, % (\url{http://www.sapienza.isc.cnr.it/}),
dell'Istituto Nazionale di Fisica Nucleare, % (\url{http://www.infn.it/}),
del Dipartimento di Fisica dell'Università Roma Tre, % (\url{http://www.uniroma3.it/}),
della Fondazione IDIS Città della Scienza, % (\url{http://www.cittadellascienza.it/}) e 
dell'Associazione Romana per le Astro-particelle (ARAP) %, \url{http://arap-astroparticelle.it/}).
e la collaborazione della EPS Rome Young Minds Section
e dell'Associazione Frascati Scienza.
\begin{center}
\fsmgraphics[height=.75cm]{}{dip-logo-uff}~\fsmgraphics[height=.75cm]{}{isc-logo}~\fsmgraphics[height=.75cm]{}{infn-logo}~\fsmgraphics[height=.75cm]{}{roma-tre}~\fsmgraphics[height=.75cm]{}{idis-logo}~\fsmgraphics[height=.75cm]{}{arap-logo-uff}\\
\end{center}

\vspace{.3em}
\begin{center}
\fsmgraphics[height=.75cm]{}{rym-logo}~\fsmgraphics[height=.75cm]{}{frascati-scienza}
\end{center}

%\begin{center}\includegraphics[width=0.2\textwidth]{isc-logo-color-it}~\includegraphcs[width=0.2\textwidth]{dip-logo-uff_w}~\includegraphics[width=0.2\textwidth]{romatre-logo-uff_w}~\includegraphics[width=0.2\textwidth]{arap-logo-uff}\end{center}

%\vspace{1cm}
%\textit{Quest'opera è rilasciata sotto la licenza Creative Commons \emph{Attribuzione-Non commerciale-Condividi allo stesso modo} 4.0 Unported. Se non specificato altrimenti, tutti gli articoli in essa contenuti sono rilasciati dai rispettivi autori sotto la medesima licenza. Per leggere una copia della licenza visita il sito web \url{http://creativecommons.org/licenses/by-nc-sa/3.0/} o spedisci una lettera a Creative Commons, 171 Second Street, Suite 300, San Francisco, California, 94105, USA.}
%\begin{center}\includegraphics[width=0.3\textwidth]{by-nc-sa}\end{center}

